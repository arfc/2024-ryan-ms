\section{\cyclus}
% fix that, too informal and introduces archetypes without context.
\cyclus is an agent-based \gls{nfc} simulator that is incredibly
versatile, one of the initial core developers (Professor Katy Huff)
likes to say that, "\cyclus can be used to model any process from making
a grilled-cheese to international nuclear fuel cycles." The software
achieves this versatility through a series of generic archetypes that are
primarily transaction based. Over the years, the user community and
developers have created a litany of nuclear specific archetypes for
everything from proliferation assessment to fuel burnup.

% discuss recipes
As \cyclus is a transactions code and not necessarily a physics code,
the reactors incorporate reactor physics through pre-defined "recipes,"
where the user specifies the isotopic concentration of the fresh and
used fuel. Users approximate the burnup of each fuel element with the
same input recipe as the same; however, in this work we incorporate a
cascading enrichment from \gls{leu+} to \gls{haleu}.
% find a citation or source that companies are actually going to do that
% (best case scenario is find it for each reactor you do it for)

% discuss EVER and CLOVER?
Novel in this work is our use of a low fidelity archetype based on the
\cycamore reactor %\cite{the summer poster}. \gls{ever}

\gls{ever} allows the user to specify multiple recipes for the fuel and
change between them at specific times.

% discuss archetypes
Users can incorporate different fuel cycle facilities into their \cyclus
simulations using generic facilities that are referred to as archetypes
in the \cyclus ecosystem. Many standard fuel cycle facilities hae been
implemented in the \cycamore repository, which holds technology agnostic
archetypes for material sources, material sinks, enrichment,
separations, and reactors.

% discuss DRE
As we have discussed, \cyclus's primary function is to keep track of
material transactions between agents. This is accomplished through the
\gls{dre}, which functions like a market where each agent brings a bid
for what and how much material they need and suppliers are matched with
buyers % cite something here.

% discuss time step stuff?
In this work we have incorporated and analyzed an alteration to the
frequency that each agent interacts with the \gls{dre} to better
understand the potential for simulation efficiency in run time and
memory usage.

\section{Metrics}
% discuss the metrics you are using
In this work we have chosen to focus on
a few key metrics to understand the
performance of each deployment scenario.
These metrics are: \gls{swu}, energy
output, mass of fuel, isotopic
composition, and reactor deployment.

\subsection{Separative Work Units}

\gls{swu}, or Separative Work Units, is a ubiquitous measure of effort
that goes into producing nuclear fuel. It is simplified as:
\begin{align}
    SWU&= Q(C_p-C_f)
    \intertext{Where:}
    SWU&= \mbox{Separative Work Units [kgSWU]}\nonumber\\
    Q&= \mbox{ Quantity of material processed [kg]}\nonumber\\
    C_p&=\mbox{ Enrichment level of the product [$\%$]}\nonumber\\
    C_f&= \mbox{ Enrichment level of the feed [$\%$].}\nonumber
\end{align}

% discuss sensitivity analysis


\subsection{Energy Output}
% just spit balling
The deployment of reactors in this work is based on energy demand, which
approximates the complicated relationship that generators and utilities
have with power expansions.

The reactors simulated herein have a static peak energy output, so the
nuance in the fleet's ability to meet the demand comes from the
deployment scheme and limitations in the fuel supply chain.

We have created a toy scenario to understand the ways in which the
deployment schemes under and over perform, and we will devote time to
discussing the realistic features of each scheme.

% sensitivity analysis?

\subsection{Mass of Fuel}

\cyclus has an understanding of the mass of material in each
transaction, in this work we will couple this tracking with an idea of
the volume of the fuel elements to get a relative sense of the volume
each scenario would produce.

We will further compare this with the storage capacity of proposed
projects in the \gls{us}--like Yucca Mountain.
% improve with comprehensive list

In addition to the volume, the mass % sensitivity analysis?

\subsection{Isotopic Composition}

The individual output composition of each reactor's fuel is
predetermined in this work through the recipes, but this metric is to
approximate fleet-wide results in each of the deployment scenarios we
have outlined.