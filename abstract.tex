%% Create an abstract that can also be used for the ProQuest abstract.
%% Note that ProQuest truncates their abstracts at 350 words.
% This is the abstract.


Understanding the nuclear fuel cycle is crucial when designing sustainable and efficient nuclear energy systems. This thesis studies timely transition scenarios for fleets of \glspl{mmr}, \glspl{xe}, and AP1000s where \gls{leup} fuel delays the demand for \gls{haleu} for the \gls{triso} fueled reactors through a greedy, random, and initially random then greedy deployment scheme to meet energy demand growths from the \gls{doe} and \gls{eia}. Using the open-source code Cyclus to model fuel cycles and the Monte Carlo code Serpent to perform neutronics calculations for the \gls{xe} and USNC \gls{mmr}, the results show that the reactor deployment scheme impacts the \gls{swu} required to meet energy demand. The greedy scheme, which prefers the highest capacity reactors, leads to the most significant increase in \gls{swu} for AP1000 \gls{leu}, while the random and initially random then greedy schemes result in more consistent increases across fuel types. By evaluating the masses of fresh and used fuel, \gls{swu}, the number of reactors, and how well each simulation meets the projected energy demand, this thesis provides a comprehensive understanding of the impact of reactor deployment schemes on the nuclear fuel cycle.

Additionally, this thesis examines the computational complexity of reactor fuel trading and removes assumptions about reactor power. The \gls{tod} reactor reduces the number of instructions in a simulation by trading fuel only when needed, while the \gls{dpr} allows for flexible power output to mirror historical or projected capacity factors. The results show that improving reactor models and simulating fuel cycle transitions leads to more efficient reactor deployment and fuel cycle design.

% Context:
% The nuclear fuel cycle plays a critical role in designing sustainable and efficient nuclear energy systems. This research focuses on enhancing fuel cycle modeling to improve reactor and fuel cycle design accuracy, with particular attention to advanced reactor technologies.

% Questions:
% This study aims to address the need for more accurate simulations of fuel cycles for advanced reactors, including the AP1000, X-Energy Xe-100, and USNC Micro Modular Reactors (MMRs). It investigates how to improve reactor core loading, fuel supply chain dynamics, and transaction modeling, while incorporating new reactor models and transition scenarios.

% Hypothesis/Prediction:
% By removing simplifying assumptions and improving simulation precision, the study hypothesizes that more accurate fuel cycle models will lead to improved reactor deployment strategies, better fuel utilization, and a more efficient nuclear fuel cycle design.

% Methods:
% Using the open-source Cyclus code, this research models fuel cycles for the AP1000, Xe-100, and MMR reactors. New reactor models—Dynamic Power Reactor (DPR), Trading On-Demand (TOD) reactor, and Enrichment Versatile Reactor (EVER)—are developed to allow flexible fuel usage. Transition scenarios, including those involving LEU+ fuel, are evaluated to assess their impact on the demand for HALEU.

% Results:
% The results show that improving reactor models and simulating fuel cycle transitions leads to more efficient reactor deployment and fuel cycle design. This work contributes to the development of advanced reactor fuel cycle simulations that provide better flexibility and accuracy in real-world energy planning.

% Conclusion/Significance:
% This research enhances existing tools for advanced reactor fuel cycle modeling, supporting more efficient reactor optimization and contributing to future nuclear energy solutions. Future work will refine these models, further expanding their use in energy planning and reactor design.


\textbf{Keywords:} Cyclus, TRISO, HALEU, LEU+, LEU Plus, Serpent, Nuclear Fuel Cycle, Memory Efficiency, Dynamic Power, Fuel Trading, Reactor Deployment, Advanced Reactors