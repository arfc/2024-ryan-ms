\section{Reactor Power and Market Interaction Conclusions and Limitations}

As Section \ref{sec:trading_reactor} identifies, the \cycamore reactor enters the Tick and Tock phases of each time step whether or not it is time to refuel. This Section uncovered that for the \gls{tod} reactor both the Tick and Tock phases were the source of less instructions than those from the \cycamore reactor. Overall, the analysis shows that the \gls{tod} reactor archetype represents an alteration in the logic of the widely used \cycamore reactor that increases the utilization, decreases the number of instructions, and maintains consistent performance with the \cycamore reactor.

This reactor was demonstrated using a 2024 test case examining the power output of the \gls{clinton}. Section \ref{sec:dpr_method} identified that the \cycamore reactor's constant power capacity results in a 3.52\% difference between the cumulative power capacity of \gls{clinton} and the \cycamore reactor modeling \gls{clinton}. The \gls{dpr} is able to replicate historic and realistic power outputs from a reactor with the only differences between \gls{dpr} and \gls{clinton} arising from floating point error below the machine epsilon.

\subsection{Future Work}
\label{sec:time_future_work}

In the future, the \gls{tod} effort can be expanded to find other ways to reduce the number of instructions germane to the simulation. Profiling revealed that the Exchange method for the \gls{dre} was routinely a larger source of instructions than the Tick and Tock methods. Outside of the reactor, incorporating tools to bypass unnecessary interactions in other fuel cycle facilities would be the next step to allowing users to develop complex purchasing agreements restricted by external factors other than material availability.
% 1. Finding other ways to reduce the number of instructions.
% 1. Investigating the Exchange method, and how the complexity can be streamlined.
% 1. Applying similar logic to other standard fuel cycle archetypes.

The \gls{dpr} is currently a stand-alone implementation of historical variation, and future work could contribute a method to generate realistic predictions of power capacity over time for the \gls{lwr} fleet. There is also a need to apply this method to creating bounding cases for the advanced reactor fleet in Section \ref{sec:reactor_models}, although each reactor model will exhibit different behavior that would require additional work to characterize. This bounding work could show the variation in power output in cases where reactors over perform their expected capacity factor, reproduce their expected capacity factor, and under perform their expected capacity factor. With this feature, the number of reactors deployed to meet demand can mirror the anticipated planning utilities will engage with. Outside of power capacity variation, the current scheme approximates the output and usage of fuel as constant over time, and implementing a similar variability in the masses and burnups of fuel, as Section \ref{sec:depletion} discusses, would strengthen the conclusions of future \cyclus simulations.


% 1. Incorporating other variations in the fuel usage (allow reactors to model hot or cold shut down).
% 1. Generating some sort of synthetic variation data based on the traditional LWR fleet performance and extrapolating that into the future.
% 1. Attempting to create bounding cases that are an analogy to the performance of the advanced reactor designs we use in the work.