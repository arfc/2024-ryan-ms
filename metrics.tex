\section{Metrics}
\label{sec:metrics}

% discuss the metrics you are using
In this work, we will develop a model of nuclear energy in the \gls{usa} using concepts from \glspl{esm} on scenarios that compare the transition from our current fleet to incorporate advanced reactor technologies not currently deployed. To compare these scenarios, we have chosen to focus on a few key metrics: \gls{swu}, energy output, mass of fuel, and reactor deployment.

\subsection{Separative Work Units}
\label{sec:swu}
% justify what swu is and why it is a valuable metric
The process of enriching uranium is a critical step in the nuclear fuel cycle, and, as we have highlighted, is expected to be a bottle neck in the deployment of advanced reactors. \gls{swu}, or a Separative Work Unit, is a ubiquitous measure of effort that goes into producing nuclear fuel. It is simplified as:
\begin{align}
    SWU&= (P \times V(x_p) + T \times V(x_t) - F \times V(x_f))\times t\nonumber
    V(x_i)&= (2 * x_i - 1) \times \ln\left(\frac{x_i}{1 - x_i}\right)\nonumber
    \intertext{Where:}
    SWU&= \mbox{Separative Work Units [kgSWU]}\nonumber\\
    P&= \mbox{Product mass flow rate [kg/d]}\nonumber\\
    F&= \mbox{Feed mass flow rate [kg/d]}\nonumber\\
    T&= \mbox{Tails mass flow rate [kg/d]}\nonumber\\
    V&= \mbox{Separation Potential [-]}\nonumber\\
    x_i&= \mbox{Weight fraction of $^{235}U$ in the i stream [-]}\nonumber\\
    x_p&= \mbox{Weight fraction of $^{235}U$ in the product stream [-]}\nonumber\\
    x_f&= \mbox{Weight fraction of $^{235}U$ in the feed stream [-]}\nonumber\\
    x_t&= \mbox{Weight fraction of $^{235}U$ in the tails stream [-]}\nonumber\\
    t&= \mbox{Time [d]}\nonumber
\end{align}

In this work, we will compare the \gls{swu} required for each scenario to understand the relative effort required to deploy the reactors and provide a stable precursor to economic calculations. As we mention in Section \ref{sec:leup}, the definition used in the literature for \gls{leup} can be tied to the upper limit on enrichment for a Category III facility. The \gls{leup} fuel, as shown in Table \ref{tab:ar_defs}, is enriched to 9.95\% $^{235}$U, which would fall under the Category III limit. The \gls{haleu} fuel would require Category II facilities to achieve the 19.75\% $^{235}$U and 15.5\% $^{235}$U enrichment for the \gls{mmr} and \gls{xe} \gls{haleu}respectively. The \gls{swu} required for each scenario will be compared to understand the relative effort required to deploy the reactors and provide a stable precursor to economic calculations.

In Table

\begin{table}
    \centering
    \caption{SWU calculation values for each fuel type}
    \label{tab:swu_vals}
    \begin{tabular}{c c}
        \hline
        \textbf{Variable} & \textbf{Value}\\
        \hline
        \gls{mmr} \gls{leup} $x_p$ & 0.0995\\
        \gls{mmr} \gls{haleu} $x_p$ & 0.1975\\
        \gls{xe} \gls{leup} $x_p$ & 0.0995\\
        \gls{xe} \gls{haleu} $x_p$ & 0.155\\
        \gls{leu} $x_p$ & 0.045\\
        $x_f$ & 0.00711\\
        $x_t$ & 0.002\\
        \hline
    \end{tabular}
\end{table}

\subsection{Energy Output}
\label{sec:energy_output}

The deployment of reactors in this work is based on energy demand, which
approximates the complicated relationship that generators and utilities
have with power expansions. The reactors simulated herein have a static peak energy output, so the nuance in the fleet's ability to meet the demand comes from the deployment scheme and limitations in the fuel supply chain. We will devote time to discussing the realistic features of each scheme. \cyclus tracks the energy output of each reactor, and we will compare that with the demand scenarios to understand the relative performance of each deployment scheme.


\subsection{Mass of Fuel}
\label{sec:mass_of_fuel}

\cyclus tracks the mass of material in each transaction, in this work we will characterize the deployment challenge ahead of us using the fresh and used fuel accumulation to show the relative mass of fuel required to deploy the reactors in each scenario. From the mass of fuel, and knowledge of the fuel design, these results can be converted into cost metrics, transportation modeling, and hypothetical repository space considerations.