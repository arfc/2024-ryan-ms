\section{Metrics}
\label{sec:metrics}

% discuss the metrics you are using
This thesis develops a nuclear energy model of the \gls{usa} using concepts from \glspl{esm} on scenarios that compare the transition from the current fleet to incorporate advanced reactor technologies not currently deployed. To compare these scenarios, I have focused on a few key metrics: \gls{swu}, energy capacity, mass of fuel, and reactor deployment.

\subsection{Separative Work Units}
\label{sec:swu}
% justify what swu is and why it is a valuable metric
The process of enriching uranium is a critical step in the nuclear fuel cycle, and is expected to be a bottleneck in the deployment of advanced reactors. The Separative Work Unit (SWU), is a ubiquitous measure of effort that goes into separating isotopes. It is simplified as:
\begin{align}
    SWU&= (P V(x_p) + T  V(x_t) - F  V(x_f)) t\nonumber \\
    \intertext{where:}
    V(x_i)&= (2x_i - 1)  \ln\left(\frac{x_i}{1 - x_i}\right)\nonumber
    \intertext{where:}
    SWU&= \mbox{Separative Work Units [kgSWU]}\nonumber\\
    P&= \mbox{Product mass flow rate [kg/d]}\nonumber\\
    F&= \mbox{Feed mass flow rate [kg/d]}\nonumber\\
    T&= \mbox{Tails mass flow rate [kg/d]}\nonumber\\
    V(x_i)&= \mbox{Separation Potential [-]}\nonumber\\
    x_i&= \mbox{Weight fraction of $^{235}U$ in the i stream [-]}\nonumber\\
    x_p&= \mbox{Weight fraction of $^{235}U$ in the product stream [-]}\nonumber\\
    x_f&= \mbox{Weight fraction of $^{235}U$ in the feed stream [-]}\nonumber\\
    x_t&= \mbox{Weight fraction of $^{235}U$ in the tails stream [-]}\nonumber\\
    t&= \mbox{Time [d]}\nonumber
\end{align}

This thesis compares the \gls{swu} required for each scenario to understand the relative effort required to deploy the reactors and provide a stable precursor to economic calculations. As mentioned in Section \ref{sec:leup}, the definition used in the literature for \gls{leup} can be tied to the upper limit on enrichment for a Category III facility. The \gls{leup} fuel, as shown in Table \ref{tab:ar_defs}, is enriched to 9.95 w\% $^{235}$U, which would fall under the Category III limit. The \gls{haleu} fuel would require Category II facilities to achieve the 19.75 w\% $^{235}$U and 15.5 w\% $^{235}$U enrichment for the \gls{mmr} and \gls{xe} \gls{haleu}. Table \ref{tab:swu_vals} shows the \gls{swu} calculation values for each fuel type.

\begin{table}[H]
    \centering
    \caption{SWU calculation values for each fuel type.}
    \label{tab:swu_vals}
    \begin{tabular}{l l}
        \hline
        \textbf{Variable} & \textbf{Value}\\
        \hline
        \gls{mmr} \gls{leup} $x_p$ & 0.0995\\
        \gls{mmr} \gls{haleu} $x_p$ & 0.1975\\
        \gls{xe} \gls{leup} $x_p$ & 0.0995\\
        \gls{xe} \gls{haleu} $x_p$ & 0.155\\
        \gls{leu} $x_p$ & 0.045\\
        $x_f$ & 0.00711\\
        $x_t$ & 0.002\\
        \hline
    \end{tabular}
\end{table}

\subsection{Energy Capacity}
\label{sec:energy_output}

The reactor deployment schemes in this thesis meet the projected energy demand scenarios from the \gls{doe} and \gls{eia} outlined in Table \ref{tab:demand_scenarios}. The reactors simulated herein have a static peak energy capacity, so the nuance in the fleet's ability to meet the demand comes from the deployment scheme and limitations in the fuel supply chain. This thesis discusses the realistic features of each scheme and compares the energy capacity with the demand scenarios to understand the relative performance of each deployment scheme.


\subsection{Mass of Fuel}
\label{sec:mass_of_fuel}

\cyclus tracks the mass of material in each transaction. This thesis focuses on the cumulative mass of fuel, both fresh an used, that comes from deploying the reactors in each scenario. From the mass of fuel and fuel design, future work can convert these results into cost metrics, transportation indicators, and hypothetical repository space considerations.