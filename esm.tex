\section{Energy System Modeling}
\label{sec:esm}

As countries like the \gls{usa} and \gls{uk} developed electrical infrastructure their approach was often centralized, which resulted from an attitude that Dieter Helm from the University of Oxford describes as being prevailing until the end of the 1970s \cite{helm_energy_2002}. Due to the heavy state involvement, energy planning is a concept that has existed for many decades now. Evidence of this contemporary category of planning can be found as far back as 1967 in nationalized industry reports from the \gls{uk} \cite{treasury_nationalised_1967}, and was a top-of-mind consideration in the \gls{usa} and other countries as well.

In 1973 Michael Posner from the University of Cambridge published his book \textit{Fuel Policy A Study In Applied Economics} \cite{posner_fuel_1973}, which set out to describe methods that large institutions could use to make decisions about energy. This book, in connection with the 1973 oil crisis, was a wake-up call for many countries. The crisis led to the development of energy planning models that could be used to evaluate the impact of different policies on energy systems as disruptions tend to do \cite{plazas_disrupt_2022}. The resulting models were used to develop long-term energy plans that could help countries increase their energy security.

Today, utilities, countries, and other organizations use \glspl{esm} to model the behavior of energy systems in different economic contexts, such as the cost of energy, the price of carbon, and the availability of financing. These contexts can focus on developing favorable conditions for new technologies, understanding the relationship between actors, predict future trends, and the impact of different policies on energy systems. Decision-makers compare the behavior of energy systems in different scenarios to a baseline, such as business-as-usual scenarios compared with low-carbon or high-renewable scenarios. These are effective across regulated, competitive, and hybrid markets. As \glspl{esm} have evolved, they have become more sophisticated and can now model the behavior of energy systems in different social contexts, such as the adoption of energy efficiency measures, the acceptance of renewable energy, and the resistance to clean energy such as the Osier tool ((((((((((cite osier when joss paper comes out)))))))))).

Although there are myriad types of \gls{esm}, the top-down and bottom-up philosophies to their construction dictate the restrictions your model will place on the type of questions you can answer. In the top-down approach, the modeler starts with a high-level view of the energy system and then drills down into the details. This approach is useful for understanding the overall behavior of the energy system and the impact of different policies on the system \cite{laha_energy_2017}. In the bottom-up approach, the modeler starts with the details of the energy system and then builds up to a high-level view. This approach is useful for understanding the behavior of individual components of the energy system and the impact of different technologies on the system \cite{ipcc_ch2_2000,laha_energy_2017}.