%% Package and Class "uiucthesis2014" for use with LaTeX2e.
\documentclass[edeposit,fullpage]{uiucthesis2018}


\usepackage[acronym,toc]{glossaries}
\newacronym{cta}{CTA}{Chicago Transit Authority}
\newacronym{usa}{USA}{United States of America}
\newacronym{ceja}{CEJA}{Climate and Equitable Jobs Act}
\newacronym{padd}{PADD}{Petroleum Administration for Defense District}
\newacronym{smr}{SMR}{Steam Methane Reformation}
\newacronym{hds}{HDS}{hydrodesulfurization}
\newacronym{esom}{ESOM}{Energy System Optimization Model}
\newacronym{hgl}{HGL}{hydrocarbon gas liquids}
\newacronym{btu}{Btu}{British thermal unit}
\newacronym{eia}{EIA}{U.S. Energy Information Administration}
\newacronym{swu}{SWU}{separative work units}
\newacronym{triso}{TRISO}{TRi-structural ISOtropic}
\newacronym{uf}{UF}{Used Fuel}
\newacronym{nfc}{NFC}{nuclear fuel cycle}
\newacronym{lwr}{LWR}{Light Water Reactor}
\newacronym{ipyc}{IPyC}{Inner PyroCarbon layer}
\newacronym{opyc}{OPyC}{Outer PyroCarbon layer}
\newacronym{htgr}{HTGR}{High-Temperature Gas-cooled Reactor}
\newacronym{uk}{UK}{United Kingdom}
\newacronym{fb-cvd}{FB-CVD}{Fluidized-Bed Chemical Vapor Deposition system}
\newacronym{bwxt}{BWXT}{BWX Technologies, Inc.}
\newacronym{usnc}{USNC}{Ultra Safe Nuclear Corporation}
\newacronym{nrc}{NRC}{U.S. Nuclear Regulatory Commission}
\newacronym{pfm}{PFM}{the Pilot Fuel Manufacturing facility}
\newacronym{tf3}{TF3}{the TRISO-X Fuel Fabrication Facility}
\newacronym{wna}{WNA}{the World Nuclear Association}
\newacronym{bwr}{BWR}{Boiling Water Reactor}
\newacronym{pwr}{PWR}{Pressurized Water Reactor}
\newacronym{hwr}{HWR}{Heavy Water Reactor}
\newacronym{gcr}{GCR}{Gas Cooled Reactor}
\newacronym{msr}{MSR}{Molten Salt Reactor}
\newacronym{lmfbr}{LMFBR}{Liquid Metal Fast Breeder Reactor}
\newacronym{rmbk}{RMBK}{graphite-moderated water-cooled reactor}
\newacronym{tru}{TRU}{transuranic isotopes}
\newacronym{mox}{MOX}{mixed oxide fuel}
\newacronym{leup}{LEU$^+$}{low enriched uranium$^+$}
\newacronym{haleu}{HALEU}{high assay low enriched uranium}
\newacronym{dre}{DRE}{Dynamic Resource Exchange}
\newacronym{ever}{EVER}{Enrichment Versatile non-Equilibrium Reactor}
\newacronym{clover}{CLOVER}{Core LOading versatile non-Equilibrium Reactor}
\newacronym{doe}{DOE}{U.S. Department of Energy}
\newacronym{pris}{PRIS}{Power Reactor Information System}
\newacronym{iaea}{IAEA}{International Atomic Energy Agency}
\newacronym{arfc}{ARFC}{Advanced Reactors and Fuel Cycles}
\newacronym{ercot}{ERCOT}{Electric Reliability Council of Texas}
\newacronym{esm}{ESM}{Energy System Model}
\newacronym{iea}{IEA}{International Energy Agency}
\newacronym{iiasa}{IIASA}{International Institute for Applied Systems Analysis}
\newacronym{bau}{BAU}{Business-as-Usual}
\newacronym{leu}{LEU}{low enriched uranium}
\newacronym{heu}{HEU}{highly enriched uranium}
\newacronym{nea}{NEA}{Nuclear Energy Agency}
\newacronym{fbcvd}{FB-CVD}{fluidized-bed chemical vapor deposition system}
\newacronym{aec}{AEC}{Atomic Energy Commission}
\newacronym{mmr}{MMR}{Micro Modular Reactor}
\newacronym{fhr}{FHR}{Fluoride Salt-Cooled, High Temperature Reactor}
\newacronym{lace}{LACE}{Los Alamos Criticality Engine}

\usepackage{xspace}
\usepackage{graphics}
\newcommand{\cycamore}{\textsc{Cycamore}\xspace}
\newcommand{\cyclus}{\textsc{Cyclus}\xspace}


\usepackage{placeins}
\usepackage{booktabs} % nice rules (thick lines) for tables
\usepackage{microtype} % improves typography for PDF

\usepackage[hyphens]{url}
\usepackage{hyperref}
\usepackage{subfig}
\usepackage{hhline}
\usepackage{amsmath}
\usepackage{color}
\usepackage{multirow}
\usepackage{siunitx}
\sisetup{
    input-decimal-markers = .,input-ignore = {,},table-number-alignment = right,
    group-separator={,}, group-four-digits = true
}
\usepackage{fourier}
\usepackage{booktabs}
\newcommand\tab[1][1cm]{\hspace*{#1}}

\usepackage{threeparttable, tablefootnote}

%tikzpicture fit to page width
\usepackage{environ}
\makeatletter
\newsavebox{\measure@tikzpicture}
\NewEnviron{scaletikzpicturetowidth}[1]{%
  \def\tikz@width{#1}%
  \def\tikzscale{1}\begin{lrbox}{\measure@tikzpicture}%
  \BODY
  \end{lrbox}
  \pgfmathparse{#1/\wd\measure@tikzpicture}%
  \edef\tikzscale{\pgfmathresult}%
  \BODY
}

\usepackage{tabularx}
\newcolumntype{b}{>{\hsize=1.0\hsize}X}
\newcolumntype{q}{>{\hsize=0.5\hsize}X}
\newcolumntype{R}{>{\raggedleft\arraybackslash\hsize=0.5\hsize}X}
\newcolumntype{z}{>{\hsize=0.75\hsize}X}
\newcolumntype{s}{>{\hsize=.5\hsize}X}
\newcolumntype{m}{>{\hsize=.75\hsize}X}

\usepackage{cleveref}
\usepackage{datatool}
\usepackage[numbers]{natbib}
\usepackage{notoccite}


\usepackage{tikz}
\usetikzlibrary{positioning, arrows, decorations, shapes}

\usetikzlibrary{shapes.geometric,arrows}
\tikzstyle{process} = [rectangle, rounded corners, minimum width=2.5cm, minimum height=1cm,text centered, draw=black, fill=blue!30]

\tikzstyle{object} = [ellipse, rounded corners, minimum width=3cm, minimum height=1cm,text centered, draw=black, fill=green!30]
\tikzstyle{objectr} = [ellipse, rounded corners, minimum width=3cm, minimum height=1cm,text centered, draw=black, fill=red!30]

\tikzstyle{empty} =  [rectangle, rounded corners, minimum width=2.5cm, minimum height=0.7cm,text centered, draw=black, fill=white!30]
\tikzstyle{arrow} = [thick,->,>=stealth]


\title{Example Template}
\author{Nathan Sean Ryan}
\department{Nuclear, Plasma, Radiological Engineering}
\schools{B.S., University of Illinois - Urbana Champaign, 2022}
\msthesis
\advisor{Madicken Munk}
\degreeyear{2024}
\committee{Assistant Professor Madicken Munk \\ Professor Kathryn D. Huff}


\begin{document}
\maketitle

\frontmatter
%% Create an abstract that can also be used for the ProQuest abstract.
%% Note that ProQuest truncates their abstracts at 350 words.
\begin{abstract}

%% Create an abstract that can also be used for the ProQuest abstract.
%% Note that ProQuest truncates their abstracts at 350 words.
This is the abstract.

\end{abstract}

\chapter*{Acknowledgments}

Acks.

%% The thesis format requires the Table of Contents to come
%% before any other major sections, all of these sections after
%% the Table of Contents must be listed therein (i.e., use \chapter,
%% not \chapter*).  Common sections to have between the Table of
%% Contents and the main text are:
%%
%% List of Tables
%% List of Figures
%% List Symbols and/or Abbreviations
%% etc.

\tableofcontents
\listoftables
\listoffigures

%% Create a List of Abbreviations. The left column
%% is 1 inch wide and left-justified
%\chapter{List of Abbreviations}
%\printglossaries
%% Create a List of Symbols. The left column
%% is 0.7 inch wide and centered

\pagebreak
\mainmatter

\chapter{Introduction}
\label{ch:introduction}
Introduction \cite{huff_extensions_2014}.

\chapter*{Appendix}
\chapter{LWRs Simulated}
\label{app:lwrs}

In this work we pull publicly available information from the \gls{pris} database to simulate the \gls{lwr} fleet in the \gls{usa}. The \gls{pris} database is a collection of information on nuclear power plants around the world, and is maintained by the \gls{iaea}. For the sake of completeness and replication of this work in Tables \ref{tab:lwr_fleet_1}, \ref{tab:lwr_fleet2}, and \ref{tab:lwr_fleet3}; we have also included the \gls{lwr} fleet that we have simulated in this work and a notebook is available on GitHub ((((((((((((cite)))))))))))) to pull the same information we used.


\begin{table}[!ht]
    \centering
    \caption{LWR Fleet Simulated, A-K}
    \label{tab:lwr_fleet_1}
    \begin{tabular}{c c c c c c c c c c}
    \hline
    \textbf{Name} & \textbf{State} & \textbf{Type} & \textbf{Vendor} & \textbf{Core size} & \textbf{Startup date} & \textbf{License} & \textbf{Retirement} & \textbf{Power cap} \\
    \hline
    Arkansas Nuclear One 1&AR & PWR & B\&W & 177 & 1974 & 2034 &      & 836.0 \\
    Arkansas Nuclear One 2&AR & PWR & CE   & 177 & 1978 & 2038 &      & 988.0 \\
    Beaver Valley 1       &PA & PWR & WE   & 157 & 1976 & 2036 &      & 908.0 \\
    Beaver Valley 2       &PA & PWR & WE   & 157 & 1987 & 2047 &      & 905.0 \\
    Big Rock Point        &MI & BWR & GE   & 84  & 1964 &      & 1997 & 67.0  \\
    Braidwood 1           &IL & PWR & WE   & 193 & 1987 & 2046 &      & 1194.0\\
    Braidwood 2           &IL & PWR & WE   & 193 & 1988 & 2047 &      & 1160.0\\
    Browns Ferry 1        &AL & BWR & GE   & 764 & 1973 & 2033 &      & 1200.0\\
    Browns Ferry 2        &AL & BWR & GE   & 764 & 1974 & 2034 &      & 1200.0\\
    Browns Ferry 3        &AL & BWR & GE   & 764 & 1976 & 2036 &      & 1210.0\\
    Brunswick 1           &NC & BWR & GE   & 560 & 1976 & 2036 &      & 938.0 \\
    Brunswick 2           &NC & BWR & GE   & 560 & 1974 & 2034 &      & 932.0 \\
    Byron 1               &IL & PWR & WE   & 193 & 1985 & 2044 &      & 1164.0\\
    Byron 2               &IL & PWR & WE   & 193 & 1987 & 2046 &      & 1136.0\\
    Callaway              &MO & PWR & WE   & 193 & 1984 & 2044 &      & 1215.0\\
    Calvert Cliffs 1      &MD & PWR & CE   & 217 & 1974 & 2034 &      & 877.0 \\
    Calvert Cliffs 2      &MD & PWR & CE   & 217 & 1976 & 2036 &      & 855.0 \\
    Catawba 1             &SC & PWR & WE   & 193 & 1985 & 2043 &      & 1160.0\\
    Catawba 2             &SC & PWR & WE   & 193 & 1986 & 2043 &      & 1150.0\\
    Clinton 1             &IL & BWR & GE   & 624 & 1987 & 2026 &      & 1062.0\\
    Columbia              &WA & BWR & GE   & 764 & 1984 & 2043 &      & 1131.0\\
    Comanche Peak 1       &TX & PWR & WE   & 193 & 1990 & 2030 &      & 1205.0\\
    Comanche Peak 2       &TX & PWR & WE   & 193 & 1993 & 2033 &      & 1195.0\\
    Cook 1                &MI & PWR & WE   & 193 & 1974 & 2034 &      & 1030.0\\
    Cook 2                &MI & PWR & WE   & 193 & 1977 & 2037 &      & 1168.0\\
    Cooper Station        &NE & BWR & GE   & 548 & 1974 & 2034 &      & 769.0 \\
    Crystal River 3       &FL & PWR & B\&W & 177 & 1976 &      & 2013 & 860.0 \\
    Davis-Besse           &OH & PWR & B\&W & 177 & 1977 & 2037 &      & 894.0 \\
    Diablo Canyon 1       &CA & PWR & WE   & 193 & 1984 & 2024 &      & 1138.0\\
    Diablo Canyon 2       &CA & PWR & WE   & 193 & 1985 & 2025 &      & 1118.0\\
    Dresden 1             &IL & BWR & GE   & 464 & 1959 & 2029 & 1978 & 197.0 \\
    Dresden 2             &IL & BWR & GE   & 724 & 1969 & 2029 &      & 894.0 \\
    Dresden 3             &IL & BWR & GE   & 724 & 1971 & 2031 &      & 879.0 \\
    Duane Arnold          &IA & BWR & GE   & 368 & 1974 & 2034 & 2020 & 601.0 \\
    Enrico Fermi 2        &MI & BWR & GE   & 764 & 1985 & 2045 &      & 1115.0\\
    Farley 1              &AL & PWR & WE   & 157 & 1977 & 2037 &      & 874.0 \\
    Farley 2              &AL & PWR & WE   & 157 & 1981 & 2041 &      & 883.0 \\
    Fitzpatrick           &NY & BWR & GE   & 560 & 1974 & 2034 &      & 813.0 \\
    Fort Calhoun          &NE & PWR & CE   & 133 & 1973 &      & 2016 & 482.0 \\
    Ginna                 &NY & PWR & WE   & 121 & 1969 & 2029 &      & 560.0 \\
    Grand Gulf 1          &MS & BWR & GE   & 800 & 1984 & 2044 &      & 1401.0\\
    Haddam Neck           &CT & PWR & WE   & 157 & 1967 &      & 1996 & 560.0 \\
    Harris 1              &NC & PWR & WE   & 157 & 1986 & 2046 &      & 964.0 \\
    Hatch 1               &GA & BWR & GE   & 560 & 1974 & 2034 &      & 876.0 \\
    Hatch 2               &GA & BWR & GE   & 560 & 1978 & 2038 &      & 883.0 \\
    Hope Creek            &NJ & BWR & GE   & 764 & 1986 & 2046 &      & 1172.0\\
    Humboldt Bay          &CA & BWR & GE   & 184 & 1962 &      & 1976 & 63.0  \\
    Indian Point 1        &NY & PWR & B\&W & 120 & 1962 & 2013 & 1974 & 257.0 \\
    Indian Point 2        &NY & PWR & WE   & 193 & 1973 & 2024 & 2020 & 998.0 \\
    Indian Point 3        &NY & PWR & WE   & 193 & 1975 & 2025 &      & 1030.0\\
    Kewaunee              &WI & PWR & WE   & 121 & 1973 & 2033 & 2013 & 566.0 \\
    \hline
    \end{tabular}
\end{table}

\begin{table}
    \centering
    \caption{LWR Fleet Simulated, L-St}
    \label{tab:lwr_fleet2}
    \begin{tabular}{c c c c c c c c c c}
    \hline
    \textbf{Name} & \textbf{State} & \textbf{Type} & \textbf{Vendor} & \textbf{Core size} & \textbf{Startup date} & \textbf{License} & \textbf{Retirement} & \textbf{Power cap} \\
    \hline
    La Crosse           & WI & BWR & AC   & 72  & 1967 &      & 1987 & 48.0  \\
    LaSalle County 1    & IL & BWR & GE   & 764 & 1982 & 2042 &      & 1137.0\\
    LaSalle County 2    & IL & BWR & GE   & 764 & 1983 & 2043 &      & 1140.0\\
    Limerick 1          & PA & BWR & GE   & 764 & 1985 & 2044 &      & 1134.0\\
    Limerick 2          & PA & BWR & GE   & 764 & 1989 & 2049 &      & 1134.0\\
    Maine Yankee        & ME & PWR & CE   & 217 & 1973 &      & 1996 & 860.0 \\
    McGuire 1           & NC & PWR & WE   & 193 & 1981 & 2041 &      & 1158.0\\
    McGuire 2           & NC & PWR & WE   & 193 & 1983 & 2043 &      & 1158.0\\
    Millstone 1         & CT & BWR & GE   & 580 & 1970 &      & 1998 & 641.0 \\
    Millstone 2         & CT & PWR & CE   & 217 & 1975 & 2035 &      & 869.0 \\
    Millstone 3         & CT & PWR & WE   & 193 & 1986 & 2045 &      & 1210.0\\
    Monticello          & MN & BWR & GE   & 484 & 1970 & 2030 &      & 628.0 \\
    Nine Mile Point 1   & NY & BWR & GE   & 532 & 1969 & 2029 &      & 613.0 \\
    Nine Mile Point 2   & NY & BWR & GE   & 764 & 1987 & 2046 &      & 1277.0\\
    North Anna 1        & VA & PWR & WE   & 157 & 1978 & 2038 &      & 948.0 \\
    North Anna 2        & VA & PWR & WE   & 157 & 1980 & 2040 &      & 944.0 \\
    Oconee 1            & SC & PWR & B\&W & 177 & 1973 & 2033 &      & 847.0 \\
    Oconee 2            & SC & PWR & B\&W & 177 & 1973 & 2033 &      & 848.0 \\
    Oconee 3            & SC & PWR & B\&W & 177 & 1974 & 2034 &      & 859.0 \\
    Oyster Creek        & NJ & BWR & GE   & 560 & 1969 & 2029 & 2018 & 619.0 \\
    Palisades           & MI & PWR & CE   & 204 & 1971 & 2031 &      & 805.0 \\
    Palo Verde 1        & AZ & PWR & CE   & 241 & 1985 & 2045 &      & 1311.0\\
    Palo Verde 2        & AZ & PWR & CE   & 241 & 1986 & 2046 &      & 1314.0\\
    Palo Verde 3        & AZ & PWR & CE   & 241 & 1987 & 2047 &      & 1312.0\\
    Peach Bottom 2      & PA & BWR & GE   & 764 & 1973 & 2053*&      & 1300.0\\
    Peach Bottom 3      & PA & BWR & GE   & 764 & 1974 & 2054*&      & 1331.0\\
    Perry 1             & OH & BWR & GE   & 748 & 1986 & 2026 &      & 1240.0\\
    Pilgrim 1           & MA & BWR & GE   & 580 & 1972 & 2032 & 2019 & 677.0 \\
    Point Beach 1       & WI & PWR & WE   & 121 & 1970 & 2030 &      & 591.0 \\
    Point Beach 2       & WI & PWR & WE   & 121 & 1971 & 2033 &      & 591.0 \\
    Prairie Island 1    & MN & PWR & WE   & 121 & 1973 & 2033 &      & 522.0 \\
    Prairie Island 2    & MN & PWR & WE   & 121 & 1974 & 2034 &      & 519.0 \\
    Quad Cities 1       & IL & BWR & GE   & 724 & 1972 & 2032 &      & 908.0 \\
    Quad Cities 2       & IL & BWR & GE   & 724 & 1972 & 2032 &      & 911.0 \\
    Rancho Seco         & CA & PWR & B\&W & 177 & 1974 &      & 1989 & 873.0 \\
    River Bend 1        & LA & BWR & GE   & 624 & 1985 & 2045*&      & 967.0 \\
    Robinson 2          & SC & PWR & WE   & 157 & 1970 & 2030 &      & 741.0 \\
    Salem 1             & NJ & PWR & WE   & 193 & 1976 & 2036 &      & 1169.0\\
    Salem 2             & NJ & PWR & WE   & 193 & 1981 & 2040 &      & 1158.0\\
    San Onofre 1        & CA & PWR & WE   & 157 & 1967 &      & 1992 & 436.0 \\
    San Onofre 2        & CA & PWR & CE   & 217 & 1982 &      & 2013 & 1070.0\\
    San Onofre 3        & CA & PWR & CE   & 217 & 1982 &      & 2013 & 1080.0\\
    Seabrook            & NH & PWR & WE   & 193 & 1990 & 2050*&      & 1246.0\\
    Sequoyah 1          & TN & PWR & WE   & 193 & 1980 & 2040 &      & 1152.0\\
    Sequoyah 2          & TN & PWR & WE   & 193 & 1981 & 2041 &      & 1139.0\\
    South Texas 1       & TX & PWR & WE   & 193 & 1988 & 2047 &      & 1280.0\\
    South Texas 2       & TX & PWR & WE   & 193 & 1989 & 2048 &      & 1280.0\\
    St. Lucie 1         & FL & PWR & CE   & 217 & 1976 & 2036 &      & 981.0 \\
    St. Lucie 2         & FL & PWR & CE   & 217 & 1983 & 2043 &      & 987.0 \\
    \hline
    \end{tabular}
\end{table}

\begin{table}
    \centering
    \caption{LWR Fleet Simulated, Su-Z}
    \label{tab:lwr_fleet3}
    \begin{tabular}{c c c c c c c c c c}
    \hline
    \textbf{Name} & \textbf{State} & \textbf{Type} & \textbf{Vendor} & \textbf{Core size} & \textbf{Startup date} & \textbf{License} & \textbf{Retirement} & \textbf{Power cap} \\
    \hline
    Summer 1            & SC & PWR & WE   & 157 & 1982 & 2042 &      & 973.0 \\
    Surry 1             & VA & PWR & WE   & 157 & 1972 & 2032 &      & 838.0 \\
    Surry 2             & VA & PWR & WE   & 157 & 1973 & 2033 &      & 838.0 \\
    Susquehanna 1       & PA & BWR & GE   & 764 & 1982 & 2042 &      & 1257.0 \\
    Susquehanna 2       & PA & BWR & GE   & 764 & 1984 & 2044 &      & 1257.0 \\
    Three Mile Island 1 & PA & PWR & B\&W & 177 & 1974 & 2034 & 2019 & 819.0 \\
    Three Mile Island 2 & PA & PWR & B\&W & 177 & 1978 & 2038 & 1979 & 880.0 \\
    Trojan              & OR & PWR & WE   & 193 & 1975 &      & 1992 & 1095.0 \\
    Turkey Point 3      & FL & PWR & WE   & 157 & 1972 & 2052*&      & 837.0 \\
    Turkey Point 4      & FL & PWR & WE   & 157 & 1973 & 2053*&      & 821.0 \\
    Vermont Yankee      & VT & BWR & GE   & 368 & 1972 & 2032 & 2014 & 605.0 \\
    Vogtle 1            & GA & PWR & WE   & 193 & 1987 & 2047 &      & 1150.0 \\
    Vogtle 2            & GA & PWR & WE   & 193 & 1989 & 2049 &      & 1117.0 \\
    Vogtle 3            & GA & PWR & WE   & 193 & 2023 & 2062 &      & 1117.0 \\
    Vogtle 4            & GA & PWR & WE   & 193 & 2024 & 2063 &      & 1117.0 \\
    Waterford 3         & LA & PWR & CE   & 217 & 1985 & 2044*&      & 1168.0 \\
    Watts Bar 1         & TN & PWR & WE   & 193 & 1996 & 2035 &      & 1157.0 \\
    Watts Bar 2         & TN & PWR & WE   & 193 & 2016 & 2055 &      & 1164.0 \\
    Wolf Creek 1        & KS & PWR & WE   & 193 & 1985 & 2045 &      & 1200.0 \\
    Yankee Rowe         & MA & PWR & WE   & 76  & 1960 &      & 1991 & 167.0 \\
    Zion 1              & IL & PWR & WE   & 193 & 1973 &      & 1997 & 1040.0 \\
    Zion 2              & IL & PWR & WE   & 193 & 1973 &      & 1996 & 1040.0 \\
    \hline
    \end{tabular}
\end{table}



\backmatter

\bibliographystyle{apalike}
\bibliography{bibliography}

\end{document}
\endinput
%%
%% End of file `thesis-ex.tex'.
