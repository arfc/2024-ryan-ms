The United States contributed more than $12.5\%$ to the total carbon emissions in 2020 \cite{european_commission_joint_research_centre_ghg_2021}. In response to growing climate concerns Illinois (and states like it) have announced myriad clean energy projects; however, when you add lenses of environmental justice and life cycle analysis, the transitions might result in displacement instead. In 2012 Richard York from the Oregon State Department of Sociology and Environmental Studies to published a cautionary study based on the 50 year histroy of alternative-energy installations to our modern grid asserting that "to displace 1 kWh of fossil-fuel electricity requires generating more than 11 kWh of non-fossil-fuel electricity," \cite{york_alternative_2012}. This conversion was based on 6 models of fossil fuel use from 1960-2009, accounting for levels of urbanization, manufacturing, age, and a variety of energy technologies.

This result challenges the attitude that there is a one-to-one relationship between energy facilities with comparable power. In 2019 York and co-author Shannon Bell further developed this idea saying that such proportional representation studies "do not focus their discussions on or graphically present the absolute quantity of energy in their assessments of purported energy transitions," \cite{york_energy_2019}. They demonstrate with Figure \ref{fig:both_plots} that the proportional representation misses that the total demand for energy has exploded since the Industrial Revolution. What may have looked like a transition in the mid-1800s from biofuels to coal is merely a displacement, and they show that the energy consumption of biofuels has increased since the early 1900s.

\begin{figure}[ht!]
  \begin{subfigure}{0.494\textwidth}
    % Code for the first plot
    \includegraphics[width=\linewidth]{proportional.jpg}
    \caption{Percentage energy}
    \label{fig:first_plot}
  \end{subfigure}%
  \begin{subfigure}{0.515\textwidth}
    % Code for the second plot
    \includegraphics[width=\linewidth]{total.jpg}
    \caption{Total energy}
    \label{fig:second_plot}
  \end{subfigure}
  \caption{Global energy consumption (exajoules) by source. 1800–2017 \cite{york_energy_2019}}
  \label{fig:both_plots}
\end{figure}

Similarly, what looks in Figure \ref{fig:first_plot} like a transition away from coal in the early 1900s, with the introduction of alternatives like oil and hydro, belies the continued increase in coal consumption into the early 2000s. If what we have axiomatically understood as a transition is not happening, we arrive at a kernel of several grand challenges to our society and our elected officials. A 2020 paper on the concept of feminist energy systems by Bell et \textit{al.} furthered the idea that "renewable energy systems do not automatically produce democracy and justice, nor are they necessarily sustainable. New fuel technologies alone, without new fuel politics, are unlikely to resolve the looming climate disaster," \cite{bell_toward_2020}. Their argument places the concept of an energy system in a socioecological, political, economic, and technological framework that will challenge policymakers to think of the way future generations will place value in each of those buckets.

The policy inertia behind the monetary valuation of our energy system is something that future generations could overcome, upending the incentives policymakers might implement to drive an actual transition instead of a displacement as we have discussed. If decision-makers focus on what a policy will do only for the term of their service, they drastically undervalue the impact that daily climate actions will have hundreds of years down the line. We see this dichotomy in the 2020 grid failings in Texas during the unseasonably cold front they experienced. From the outside, we can see how an extreme weather event would create a great demand. This case is a microcosm of a drastic change in the values a society had in an energy grid. In the span of a couple of weeks a state that had proudly touted its achievement of a sustained grid \cite{texas_ercot_nodate} experienced massive failures only for its service to resume.

We can legislate for such changes by developing policy frameworks that bring in the community and ask for continuous feedback. Elisa Papadis and George Tsatsaronis set out to update the vision for a well-designed policy package in their 2020 paper, surmising that producing policy "with measures such as carefully introduced targeted investment subsidies, performance standards and mandates, communication and education campaigns and a CO$_2$ tax for global aviation and shipping" constitutes achieving this legislative framework \cite{papadis_challenges_2020}. This approach will require geographically bespoke solutions that draw in stakeholders, and keep them in a perpetual feedback cycle where their changing values are reflected in updates. They go on to advocate for expansion and investment in the massively complex \gls{usa} power grid due to the requirements of more flexibly generated capacity.

Flexibility is a seemingly ubiquitous goal of decarbonized industries, like chemical producers, which highlight big emitters that are large volume/low-profit goods (disincentivizing development) \cite{mallapragada_decarbonization_2023} or farm researchers who highlight the growing importance of human intervention as climate change impacts their crop in a negative feedback cycle \cite{farokhi_soofi_farm_2022}. The start is focusing our efforts where investment can have the largest impact in the shortest time, and to consult the changing valuation of stakeholders in how we deploy electrification and updates to the grid.