% Provide an introduction to the literature review. Explain the scope of the
% review.
In this section, we will discuss the transition scenarios for the deployment of new nuclear reactors in the \gls{usa} and the reactor models used in this work. We will also discuss the theoretical framework and key concepts that underpin the research, and the results for each deployment scheme.

\section{Reactor Models}
\label{sec:reactor_models}

In this work, we will explore various transition scenarios for the deployment in the \gls{usa} of the: 1) X-Energy Xe-100-like \glspl{htgr}; 2) \gls{usnc} \gls{mmr}-like \glspl{htgr}; and 3) and Westinghouse AP1000-like \glspl{pwr}. To accommodate our assumption that the \gls{haleu} fueled \gls{triso} reactors will first accept \gls{leup} fuel, we have adapted Bachmann's \gls{mmr}-like Serpent model \cite{bachmann_mmr_like_2023} to accept \gls{leup} fuel. These reactor models are not based on confidential or proprietary information, they are constructed from publicly available data to approximate the aggregate behavior.

\begin{table}[htbp]
   \centering
   \caption{Advanced reactor design specifications.}
   \label{tab:ar_defs}
   \begin{tabular}{c c c c}
      \hline
      \textbf{Design Criteria} & \textbf{MMR-Like} \cite{usnc_design_2021} & \textbf{Xe-100-Like} \cite{nuscale_chapter_2018} & \textbf{AP1000} \\
      \hline
      Reactor type & \gls{htgr} & \gls{htgr} & \gls{pwr} \\
      Power Output [MWth] & 15 & 100 & 1000 \\
      Fuel Type & \gls{triso} & \gls{triso} & UO$_2$ \\
      Enrichment [\% $^{235}$U] & 9.95, 19.75 & 9.95, 15.5 & 5 \\
      Cycle Length [yrs] & 20 & Online Refuel & 18 \\
      Discharge Burnup [GWd/MTU] & 82 & 168 & 65 \\
      Reactor Lifetime [yrs] & 20 & 60 & 60 \\
      \hline
   \end{tabular}
\end{table}

In the following subsections, we will discuss the reactors in greater detail. One limitation of this work that we will devote future research to is the approximation for the \gls{leup} fueled version of the reactors as achieving the same burnup and power level as the \gls{haleu} fueled version. This assumption is not necessarily valid, and we will explore the implications of this assumption in future work. However, we expect that, due to the limited deployment of \gls{leup} fuel, the discrepancies will be well-understood.

\subsection{MMR-like Reactor}
\label{sec:mmr}

The \gls{usnc} \gls{mmr} is a \gls{htgr} that uses \gls{triso} fuel. The reactor has a thermal output of 15 MW and a cycle length of 20 years. The fuel is enriched to 9.95\% $^{235}$U and has a discharge burnup of 82 GWd/MTU. The reactor has a lifetime of 20 years. The reactor in this work is an approximation based on publicly available data and is not based on confidential or proprietary information. The model was originally developed by Bachmann et al. (((((((((cite mmr zenodo))))))))), and is implemented in this work as-is for the \gls{haleu} version of the reactor while the \gls{leup} version is adapted from the \gls{haleu} version.

As of this writing, there is an active partnership between \gls{uiuc} and \gls{usnc} to deploy their \gls{mmr} in the pre-licensing phase with the \gls{nrc} and is expected to be operational in the 2030s.


\subsection{Xe-100-like Reactor}
\label{sec:xe}

The X-Energy Xe-100 is a \gls{htgr} that uses \gls{triso} fuel. The reactor has a thermal output of 100 MW and uses online refueling. The fuel is enriched to 9.95\% $^{235}$U and has a discharge burnup of 168 GWd/MTU. The reactor has a lifetime of 60 years. The reactor in this work is an approximation based on publicly available data and is not based on confidential or proprietary information. The model was originally developed by Richter et al. (((((((((cite xe zenodo))))))))), and is implemented in this work as-is for the \gls{haleu} version of the reactor while the \gls{leup} version is adapted from the \gls{haleu} version.

As of this writing, X-Energy has received a grant from the \gls{doe} to deploy their Xe-100 in the pre-licensing phase with the \gls{nrc} for projects in Texas and Washington and is expected to be operational in the 2030s. There are similar projects in early stages in Canada and the \gls{uk}.


\subsection{AP1000 Reactor}
\label{sec:ap}

The Westinghouse AP1000 is a \gls{pwr} that uses UO$_2$ fuel. The reactor has a thermal output of 1000 MW and a cycle length of 18 years. The fuel is enriched to 5\% $^{235}$U and has a discharge burnup of 65 GWd/MTU. The reactor has a lifetime of 60 years. The reactor in this work is an approximation based on publicly available data and is not based on confidential or proprietary information. As this work does not anticipate \gls{leup} being used in the AP1000, there is no such neutronics model of the reactor and we have adapted the generic \cycamore reactor archetype to represent the AP1000.

As of this writing, the AP1000 is operational in the \gls{usa} and China, and there are plans to deploy the reactor in the \gls{uk} and India.

\section{Transition Scenarios}
\label{sec:transition_scenarios}
% Discuss the theoretical framework and key concepts that underpin the research.

As there are various definitions for each class of enrichment, we will use the following for the purposes of this work:

\begin{table}[htbp]
   \centering
   \caption{Enrichment levels and their ranges.}
   \label{tab:enrichment_levels}
   \begin{tabular}{c c}
      \hline
      \textbf{Enrichment Level} & \textbf{Range [\%  $^{235}$U]} \\
      \hline
      Natural & < 0.711 \\
      \gls{leu} & 0.711-5 \\
      \gls{leup} & 5-10 \\
      \gls{haleu} & 10-20 \\
      \gls{heu} & $\geq$ 20  \\
      \hline
   \end{tabular}
\end{table}


% Identify gaps or limitations in the existing literature. Discuss areas where further research is needed.

