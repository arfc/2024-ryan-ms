% Provide an introduction to the literature review. Explain the scope of the
% review.
In this section, we will discuss the transition scenarios for the deployment of new nuclear reactors in the \gls{usa} and the reactor models used in this work. We will also discuss the theoretical framework and key concepts that underpin the research, and the results for each deployment scheme.

\section{Reactor Models}
\label{sec:reactor_models}

In this work, we will explore various transition scenarios for the deployment in the \gls{usa} of the: 1) X-Energy Xe-100-like \glspl{htgr}; 2) \gls{usnc} \gls{mmr}-like \glspl{htgr}; and 3) and Westinghouse AP1000-like \glspl{pwr}. To accommodate our assumption that the \gls{haleu} fueled \gls{triso} reactors will first accept \gls{leu+} fuel, we have adapted Bachmann's \gls{mmr}-like Serpent model \cite{bachmann_mmr_like_2023} to accept \gls{leu+} fuel. These reactor models are not based on confidential or proprietary information, they are constructed from publicly available data to approximate the aggregate behavior.

\section{Transition Scenarios}
\label{sec:transition_scenarios}
% Discuss the theoretical framework and key concepts that underpin the research.

As there are various definitions for each class of enrichment, we will use the following for the purposes of this work:

\begin{table}[htbp]
   \centering
   \caption{Enrichment levels and their ranges.}
   \label{tab:enrichment_levels}
   \begin{tabular}{c c}
      \hline
      \textbf{Enrichment Level} & \textbf{Range [\%  $^{235}$U]} \\
      \hline
      Natural & < 0.711 \\
      \gls{leu} & 0.711-5 \\
      \gls{leu+} & 5-10 \\
      \gls{haleu} & 10-20 \\
      \gls{heu} & $\geq$ 20  \\
      \hline
   \end{tabular}
\end{table}


% Identify gaps or limitations in the existing literature. Discuss areas where further research is needed.

