\section{\cyclus}
\label{sec:cyclus}
\cyclus \cite{huff_cyclus_intro_2016} is an agent-based \gls{nfc} simulator that is versatile, open-source, and modular. The software achieves this versatility through a series of generic archetypes that are primarily transaction-based. Over the years, the user community and developers have created a litany of nuclear-facility archetypes for everything from proliferation assessment to fuel burnup. Many standard fuel cycle facility archetypes have been implemented in the \cycamore repository \cite{Carlsen_cycamore_2014}, which holds technology-agnostic archetypes for material sources, material sinks, enrichment services, separations capabilities, storage services, and a generic reactor.

The architecture of \cyclus is designed to facilitate the simulation of complex interactions between various agents in a nuclear fuel cycle. Figure \ref{fig:cyclus_architecture} provides an overview of the architecture, highlighting the core components of the \cyclus kernel and the Agent API. The kernel manages the simulation's overall state and handles the scheduling and execution of events, while the Agent API allows users to define custom agents (e.g., regions, institutions, and facilities) that can interact with each other through the \gls{dre}. Facilities are the primary agents in the simulation, and they can represent various types of nuclear facilities, such as reactors, enrichment plants, and waste repositories. Each facility can have its own internal logic for managing resources and making decisions based on the interactions with other agents in the simulation. The interactions between facilities can be modified by grouping them in institutions, which are analogous to real-world organizations that manage multiple facilities. The interactions between institutions can be modified by grouping them into regions, which can represent larger geographical areas or regulatory jurisdictions. This hierarchical structure allows for a more organized and manageable way to model complex fuel cycles with multiple layers of interaction.

\begin{figure}[H]
    \centering
    \begin{tikzpicture}
        % Main rectangle
        \draw[thick, fill=lightpurple] (0,0) -- (1,1.5) -- (7, 1.5) -- (8,0) -- (8,3) -- (0,3) -- (0,0);
        \node at (4,2) {\textbf{Cyclus Kernel}};

        % Trapezoid cut-out at the bottom
        \draw[thick, fill=lightgreen] (0,-0.5) -- (1,1) -- (7,1) -- (8,-0.5) -- (6.75, -0.5) -- (6, 0.4) -- (5.25, -0.5) -- (4.75, -0.5) -- (4.75, 0.4) -- (3.25, 0.4) -- (3.25, -0.5) -- (2.75, -0.5) -- (2.75, 0.0) -- (2, 0.4) -- (1.25, 0.0) -- (1.25, -0.5) -- cycle;
        \node at (4,0.7) {\textbf{Agent API}};
        \node at (2, -0.75) {\textbf{Regions}};
        % \node at (2, -1.1) {\textbf{API}};

        \node at (4, -0.75) {\textbf{Institutions}};
        % \node at (4, -1.1) {\textbf{API}};

        \node at (6, -0.75) {\textbf{Facilities}};
        % \node at (6, -1.1) {\textbf{API}};
    \end{tikzpicture}
    \caption{\cyclus architecture overview. The Cyclus kernel manages the simulation, while the Agent API allows for the creation of regions, institutions, and facilities. Facilities are the primary agents that interact with each other through the Dynamic Resource Exchange (DRE).}
    \label{fig:cyclus_architecture}
\end{figure}

\cyclus treats each facility as an agent that can interact with other agents in
the simulation. These agents are defined by their capabilities and the resources
they can provide or consume. The agents are connected through a unique market
mechanism called the \gls{dre} that allows agents to request
resources and respond to requests from other agents. The \gls{dre} is responsible for matching resource requests with offers from suppliers and ensuring that the resources are exchanged.

Commodities, in the parlance of the \cyclus ecosystem, are passed by agents through the \gls{dre} in recorded transactions. A commodity can be anything, from raw materials (like uranium ore) to contextual concepts (e.g., money or permits). The transactions are recorded in a database that can be queried to determine the flow of materials through the simulation. As Huff et al. outline in their 2016 paper \cite{huff_cyclus_intro_2016}, treating facilities and materials independently allows for flexibility in the level of fidelity for both.

% discuss recipes
As \cyclus is a transactions code and not necessarily a physics code,
the reactors incorporate physics through pre-defined \textit{recipes},
where the user specifies the concentration of isotopes in a fuel form.
Pre-defining recipes can reduce the precision with which
fuel compositions are calculated, but it dramatically increases computational
efficiency for simulations of complex fuel cycles. For example, depending
on how fuel is discharged from a core at the end of a reactor's lifetime, there
are instances when the model will over or under-predict the concentration of
isotopes, thereby injecting a level of error into calculations.
Users approximate the burnup of each fuel element with the same input recipe to
be the same; however, this thesis incorporates a cascading enrichment from
\gls{leup} to \gls{haleu}; \gls{leup} in the short term, while reactor vendors
and fuel suppliers work with the government to establish the supply chain for
\gls{haleu}.

% discuss EVER and CLOVER?
% Novel in this work is our use of a low fidelity archetype based on the \cycamore reactor \gls{ever}, which allows the user to specify multiple recipes for the fuel and change between them at specific times.

% discuss region, institution, and facility relationships
In \cyclus, the user defines the simulation by specifying the regions,
institutions, and facilities that will be present. Regions
are collections of institutions, where institutions are collections of
facilities, while facilities are the primary agents interacting in the
simulation. The user can define the relationships between regions,
institutions, and facilities to model the flow of materials and resources
through the simulation. Through initial conditions, the user can tailor their
simulation for any historical or imagined starting point based on the expected
facilities and resources available from the outset.

\subsection{Fuel Depletion}
\label{sec:depletion}

Fuel depletion is a critical aspect of fuel cycle simulations, as it directly
influences the characteristics and behavior of \gls{uf}. The composition of
this fuel, shaped by factors like decay heat, the quantity of fissile material,
and the volume of \gls{uf}, has profound implications for various stages of the
nuclear fuel cycle. The effects of depletion on these properties are the
thermal and physical characteristics of the fuel, as well as the practical
considerations such as the transportation of \gls{uf}, the limits of repository
storage, and the potential for reprocessing and recycling of materials. Thus,
any comprehensive fuel cycle simulation must account for fuel depletion,
ensuring that the resulting data reflect realistic conditions and constraints.

In many fuel cycle simulators, fuel depletion is managed using pre-defined
compositions, which allow for rapid calculations and straightforward modeling
of fuel cycles. Tools such as VISION \cite{yacout_visionverifiable_2006}, the
CYCAMORE Reactor archetype in \cyclus--which uses the aforementioned
\textit{recipes}--and ORION utilize this methodology. In these frameworks, the
compositions of \gls{uf} are established in advance and derived from
separate depletion modeling efforts. This approach is particularly effective
for once-through fuel cycles, where fresh fuel compositions remain constant
across refueling periods, leading to minimal variation in the characteristics
of discharged fuel. While this method offers simplicity and speed, it may not
capture the nuanced behaviors seen in more complex fuel cycles, where dynamic
changes in fuel composition are more pronounced.

Users have developed several archetypes to further expand the capabilities of \cyclus in modeling fuel depletion. Bright-lite \cite{schneider_integrated_2016}, for instance, introduces a comprehensive
framework for evaluating fuel compositions based on burnup and criticality.
This archetype offers two operational modes—forward and blending mode—allowing
users to tailor depletion modeling to specific scenarios. The initial recipe is depleted based on a given fluence in forward mode. In blending mode, the
reactor is connected to a fabrication facility that mixes material streams to
meet a burnup criticality or conversion ratio. The burnups and material
definitions need to be given, which is the first step in implementing this
archetype. CyBORG \cite{skutnik_cyborg_2016}, another \cyclus archetype,
integrates \cyclus with ORIGEN by generating a problem-specific cross section
library (which it then feeds to ORIGEN to perform a single depletion
calculation for the core), enabling a more nuanced approach to modeling fuel
cycles. Although CyBORG requires access to the export-controlled ORIGEN, it
enhances the accuracy of \gls{uf} compositions. The ann\_pwr \cite{bae_deep_2020} archetype employs neural networks trained on
historical data to predict fuel compositions based on burnup and initial
enrichment. While achieving results with less than 1\% error 0.23\% of the
time as ORIGEN, ann\_pwr's applicability is limited to \gls{pwr} designs, highlighting a need for broader models that can encompass diverse
reactor types.

Dynamic modeling of fuel depletion represents an evolution in fuel cycle simulations, allowing for real-time updates to fuel compositions as material properties evolve. This approach is crucial for accurately reflecting the
influences of fuel depletion on material properties. Various simulators outside
the \cyclus ecosystem, including ORION \cite{feng_standardized_2016}, DYMOND
\cite{richards_application_2021}, and NFCSim \cite{schneider_nfcsim_2005}, are capable of dynamic modeling. For instance, ORION allows users to
define material compositions using recipes and autonomously model decay and
depletion. DYMOND enhances accuracy by coupling with ORIGEN2, enabling
criticality searches that refine fresh fuel compositions based on updated
\gls{uf} data. NFCSim's coupling with the \gls{lace} further exemplifies this
trend by employing fluence-dependent calculations to ascertain the evolving
nature of nuclear materials. These advancements in dynamic modeling are
essential for improving the fidelity and reliability of fuel cycle simulations,
particularly as the nuclear industry moves toward more intricate and
sustainable fuel management strategies.

The OpenMCyclus archetype \cite{openmcyclus_paper} enhances the \cyclus
ecosystem by introducing an open-source real-time fuel depletion tool for
\cyclus simulations, building off of the concept of CyBORG. Unlike traditional
approaches that rely on pre-defined recipes for fuel compositions, OpenMCyclus
integrates with OpenMC \cite{romano_openmc_2015} to dynamically update spent
fuel compositions throughout the simulation, allowing for greater accuracy and
flexibility in modeling various reactor designs. This real-time depletion
capability is valuable for assessing the impacts of different fuel cycle
strategies, as it accommodates changes in fuel composition and operational
conditions without the need for restrictive licensing agreements associated
with other depletion tools.


% We aim to provide the most versatile and powerful foundation
% for \cyclus users by focusing on these core toolkit capabilities. This will
% enable researchers and developers to create more accurate and detailed
% simulations of \glspl{nfc}, ultimately allowing for better decision-making and
% policy development in the nuclear industry. We have identified the appropriate
% policy to update in material transactions and leverage the existing dormant
% purchasing policy to create a robust implementation in the \cycamore
% archetypes, ensuring that the most users possible will benefit. By implementing
% these improvements, we aim to ensure that the \cyclus toolkit remains at the
% forefront of \gls{nfc} simulation, providing a robust and flexible platform for
% users to explore and analyze various scenarios and strategies.