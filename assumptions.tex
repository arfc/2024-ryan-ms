\begin{itemize}
    \item The demand increase assumes that nuclear energy's share of generation remains constant over time. When we say a 15$\%$ increase in demand, we mean a 15$\%$ increase in nuclear energy generation (as the EIA numbers are meant to reflect the total energy demand, this conversion is only possible by assuming that the percentage of nuclear capacity is the same). This assumption is not reflected in the demand scenarios from the \gls{doe} liftoff report, which are specific to nuclear deployment increases and the number is agnostic to the total increase.
    \item \gls{lwr}s have an 18 month cycle with no deviation
    \item \gls{lwr}s have a 1 month outage
    \item \gls{lwr}s have an assembly size of 427.38589211618256
    \item \gls{lwr}s have a batch size of 80
    \item All reactors have constant power output when not in outage
    \item No new \gls{lwr}s are built after 2024
    \item 
\end{itemize}