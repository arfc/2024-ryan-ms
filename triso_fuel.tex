\section{TRISO Fuel}
\label{sec:triso_fuel}

In this work, we adapt the approach of Bachmann et \textit{al.} \cite{bachmann_enrichment_2021} to focus on \gls{triso} fueled reactor designs alongside traditional fuel forms at various enrichments. \gls{triso} is not a classification of enrichment, and as such, there are several reactors that make use of different fuel enrichment. As there are various definitions for each class of enrichment, we will use the following for the purposes of this work:

\begin{table}[htbp]
   \centering
   \caption{Enrichment levels and their ranges.}
   \label{tab:enrichment_levels}
   \begin{tabular}{c c}
      \hline
      \textbf{Enrichment Level} & \textbf{Range [\%  $^{235}$U]} \\
      \hline
      natural uranium & < 0.711 \\
      \gls{leu} & 0.711-5 \\
      \gls{leu+} & 5-10 \\
      \gls{haleu} & 10-20 \\
      \gls{heu} & $\geq$ 20  \\
      \hline
   \end{tabular}
\end{table}

The \gls{triso} fuel fabrication process is a complex process that involves multiple steps. We have outlined the general process of producing the fuel in Section \ref{sec:nfc}, here we will distinguish \gls{triso} from the traditional metalic fuels used in \glspl{lwr}.


The process begins with the production of \gls{triso} particles, which are then coated with multiple layers of pyrolytic carbon and silicon carbide. The coated particles are then loaded into fuel compacts, which are then loaded into fuel elements. The fuel elements are then loaded into fuel assemblies, which are then loaded into the reactor core.