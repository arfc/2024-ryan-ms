\section{LEU Plus}
\label{sec:leup}

In 2020, a \gls{haleu} workshop report led by Monica Regalbuto \cite{regalbuto_high_assay_2020} highlighted the unique regulatory challenges of establishing a \gls{haleu} fuel cycle in the \gls{usa}. It noted that part of enriching \gls{haleu} is first to produce \gls{leup}, defined as between 5\% and 10\% $^{235}U$ enrichment. The report notes that \gls{leup} facilities would fall under a similar category of regulations as our existing \gls{leu} fuel cycle, allowing existing enrichment servicers to leverage their experience and infrastructure before taking on the increased regulatory burden of producing \gls{haleu}. If a reactor could be redesigned to accommodate it, using \gls{leup} could delay the demand for \gls{haleu}. Table \ref{tab:enrichment_levels} shows the various levels of enrichment for uranium that we will use in this work.

\begin{table}[H]
   \centering
   \caption{Enrichment levels and their ranges.}
   \label{tab:enrichment_levels}
   \begin{tabular}{c c}
      \hline
      \textbf{Enrichment Level} & \textbf{Range [\%  $^{235}$U]} \\
      \hline
      Natural & < 0.711 \\
      \gls{leu} & 0.711-5 \\
      \gls{leup} & 5-10 \\
      \gls{haleu} & 10-20 \\
      \gls{heu} & $\geq$ 20  \\
      \hline
   \end{tabular}
\end{table}

For a fuel cycle containing \gls{leup}, one of the primary advantages is that the facility to produce it would fall under the same licensing category as \gls{leu} fuel. The \gls{nrc} defines a \textit{special nuclear material of low strategic significance} as meeting one of three criteria, the most notable of which for our purposes is "(3) 10,000 grams or more of uranium-235 (contained in uranium enriched above natural but less than 10 percent in the U–235 isotope)," \cite{nrc_catiii}. This facility definition is where the upper limit of the \gls{leup} range arises.

To enrich to up to \gls{haleu}, facilities such as TRISO-X LLC and Kairos Power Atlas Fuel Fabrication Facility must move up a category to \textit{special nuclear material of moderate strategic significance} (Category II). Thus, \gls{leup} is an attractive intermediary step for servicers wishing to minimize the size of a Category II facility (thereby reducing costs) as it is the same category we have historically licensed for \gls{leu} fuel enrichment.

Traditional \glspl{lwr} could receive benefits from using \gls{leup} fuel; as outlined by L\'{o}pez-Luna et al. \cite{24_month_cycle_bwr}, incorporating such fuel rods would allow for a 24-month cycle in the \gls{bwr} design they studied and would reduce the levelized cost of the nuclear fuel cycle they simulated. In October 2024, Framatome announced that their 6 wt$\%$ $^{235}$U GAIA fuel assemblies completed their third 18-month fuel cycle at the Vogtle plant in Georgia \cite{framatome_press_2024}, with the eventual goal of this process being commercialization of new accident tolerant fuels that can potentially support \gls{leup}. The growing body of work indicates that \gls{leup} fuels could suite a variety nuclear of technologies; however, the fuel does not exist in a vacuum.

Increased prevalence of higher enrichment fuels will require modifications to the existing supply chain, particularly to ensure the continued safety of workers and the public. A 2022 report from Shaw and Clarity out of \gls{ornl} highlighted that existing nuclear fuel vault configurations at \glspl{bwr} and \glspl{pwr} did not have sufficient margins to satisfy regulatory requirements when fully flooded \cite{leup_atf_storage_impacts}. Their report only studied the impacts of 6.5 wt$\%$ and 8 wt$\%$ fuel, but they conclude their report noting that \gls{haleu} fuel would similarly require significant changes to existing fuel storage infrastructure.